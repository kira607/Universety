\usepackage{xcolor}
\usepackage{listings}  % листинги кода из файлов

\definecolor{codegreen}{rgb}{0,0.6,0}
\definecolor{codegray}{rgb}{0.5,0.5,0.5}
\definecolor{codepurple}{rgb}{0.58,0,0.82}
\definecolor{backcolour}{rgb}{0.95,0.95,0.92}

\lstdefinestyle{cpp}
{
	backgroundcolor=\color{backcolour},% цвет фона подсветки
	commentstyle=\color{codegreen},
	keywordstyle=\color{magenta},
	numberstyle=\small\color{codegray},% размер шрифта для номеров строк
	stringstyle=\color{codepurple},
	basicstyle=\ttfamily\footnotesize, % размер и начертание шрифта для подсветки кода
	breakatwhitespace=false,           % переносить строки только если есть пробел  
	breaklines=true,                   % автоматически переносить строки (да\нет)  
	captionpos=t,                      % позиция заголовка вверху [t] или внизу [b]
	keepspaces=true,                 
	numbers=left,                   % где поставить нумерацию строк (слева\справа)            
	numbersep=5pt,                  % как далеко отстоят номера строк от подсвечиваемого кода           
	showspaces=false,               % показывать или нет пробелы специальными отступами       
	showstringspaces=false,         % показывать или нет пробелы в строках
	showtabs=false,                 % показывать или нет табуляцию в строках        
	tabsize=2,                      % размер табуляции по умолчанию равен 2 пробелам
	language=c++,                   % выбор языка для подсветки (здесь это С++)                   
	stepnumber=1,                   % размер шага между двумя номерами строк
	frame=false,                    % рисовать рамку вокруг кода
	escapeinside={\%*}{*)},         % если нужно добавить комментарии в коде
	extendedchars=\true
}

\lstset{style=cpp}