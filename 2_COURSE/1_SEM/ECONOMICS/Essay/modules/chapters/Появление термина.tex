\section{Появление термина}

Самозанятость в РФ как вид занятости появилась в 2018 году вместе с Федеральным законом "О проведении эксперимента по установлению специального налогового режима "Налог на профессиональный доход" в городе федерального значения Москве, в Московской и Калужской областях, а также в Республике Татарстан (Татарстан)" от 27.11.2018 N 422-ФЗ \cite{law-self-emp}.

Этот термин обозначает некую новую форму получения вознаграждения на свой профессиональный труд,
который в свою очередь является конкретизацией функционального труда,
образующий широкую профессиональную структуру и предусматривающий 
отсутствие трудового договора и конкретного работодателя.

То есть если человек продаёт старый шарф, 
доход с продажи не попадает под категорию налогообложения для самозанятых,
так как он не занимается профессиональным трудом.

Следует отметить, 
что самозанятость предусматривает новый вид налогообложения на физическое лицо.

Было выявлено, что самозанятый человек получает свой заработок от профессионального труда не по трудовому договору, работая в найм, а от заказчиков. Простыми словами, профессиональный труд предполагает некую специализацию человека, например, маркетолог, продавец, веб-дизайнер и другие. Уже с октября этого года данный режим налогообложения действует по всей России для физических лиц с \textbf{16} лет.