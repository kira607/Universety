\section*{Заключение}
\addcontentsline{toc}{section}{Заключение}

Режим самозанятости --- не обязанность, а право. 
Право и возможность для людей выполнять свою конституционную обязанность, 
то есть платить налоги и сборы. 

Самозанятые платят налог по ставке 
4 \% при получении дохода от физических лиц и 
6 \% при получении дохода от юридических лиц и ИП. 

Согласно статье 146 Бюджетного кодекса РФ 
часть уплаченного налога зачисляется в фонд медицинского страхования (37 \%), 
а часть (63 \%) — в региональный бюджет. 

Однако пенсия складывается из стажа (от 15 лет) 
и баллов (от 30, а это --- фиксированные взносы), 
соответственно, самозанятый может рассчитывать 
только на минимальную социальную пенсию почти что равной той, 
если бы он не подтверждал налоговый режим. 
В итоге, добровольно оформленная самозанятость имеет лишь 
1 плюс для граждан --- чувство "безопасности"\ благодаря узакониванию получения дохода, 
не более. 

Для государства эффективность более, чем положительная, 
так как помимо очередного сбора налогов, 
формируется база из тех, 
кто ранее был вне поле видения, 
то есть получал "серый"\ доход.
