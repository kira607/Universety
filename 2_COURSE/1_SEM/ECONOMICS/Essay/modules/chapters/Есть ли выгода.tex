\section{Есть ли выгода?}

Вернемся к гражданину А. 

Узнав о существующем режиме налогообложения, 
он задался вопросом --- 
а для чего мне платить налоги государству за свою подработку?
Ведь я потому и подрабатываю, 
что мне не хватает дохода от преподавания в государственной школе!
С меня ведь и так налоги берут!

Поговорим о выгодах и о ставке налогообложения при становлении на учет самозанятого.

Во-первых, действует правило: 
получил доход 
(фактическая дата поступления денежных средств 
на счета налогоплательщика в банках 
либо по его поручению на счета третьих лиц) --- 
заплати налог в конце месяца 
(не позднее 25 числа нового месяца), 
ничего не получил --- можешь не платить. 

То есть, в отличие от ИП, 
где фиксировано нужно платить $ n $-нную сумму государству, 
самозанятый платит по факту получения дохода от профессиональной деятельности. 

Приятный бонус: даже, если за целый год физическое лицо, 
оформленное как самозанятое, ничего не заработал, 
он участвует в системе обязательного медицинского страхования 
и может получать помощь в медицинских учреждениях бесплатно.

Во-вторых, самозанятые освобождаются от стандартной налоговой ставки (НДФЛ) в 13\% 
с профессионального дохода, 
ставка понижается до 4\% от реализации услуг и товаров (простыми словами, выручка). 
Налог автоматически списывается со счета банковской карты, 
если лицо ранее направляло в банк согласие о подобном списании. 
В иных случаях сумму к уплате можно найти в личном кабинете сайта 
Госуслуги или в приложении Мой налог. 

Штрафы для самозанятых также предусмотрены.
Например, если гражданин нарушил условия выставления чека, 
то есть нарушил сроки формирования чека или самой суммы, 
будет взыскиваться штраф в размере 20 \% от суммы, 
на которую не был выставлен чек 
(в электронных приложениях автоматически формируются чеки, 
которые по закону необходимо направить лицу, 
оплатившему услугу/товар, 
данные о легализации дохода будут передаваться автоматически в налоговые органы. 

Если в течение 6-ти месяцев самозанятый снова нарушил условия выставления чека,
ему полагается штаф, равный всей сумме, на которую не был сформирован чек.

В-третьих, как было указано ранее, гражданин в статусе самозанятого 
может работать официально по трудовому договору или ГПХ, 
и более того, также может открыть ИП. 

Интересный факт, что немало ИП и других организаций 
специально переводят своих сотрудников на режим самозанятости, 
что уменьшить свои расходы, 
однако это не соответствует понятию самозанятости 
и налоговые в 2021 году обещают выполнить масштабную проверку. 

В-четвёртых, для подростков до 18-ти лет к стандартному вычету в 10 000 рублей может добавиться 12 130 рублей, если они зарегистрируют себя как самозанятые после 1 января 2021, будучи несовершеннолетними.

С учётом всего вышесказанного может показаться,
что самозанятость --- очень эффективный способ
работать на себя по "белой"\ схеме при этом не 
будучи сильно ограниченным огромным количеством
налогов. Не жёсткие условия труда и 
всего 4 \% с доходов, которые нужно вовремя 
отдавать налоговой. Но есть одно \textbf{НО} \ldots