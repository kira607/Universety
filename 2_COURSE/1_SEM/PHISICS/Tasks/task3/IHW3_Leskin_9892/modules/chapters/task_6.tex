
\begin{center}
    \textbf{Задача 3.5 --- 1}
\end{center}

\underline{\textbf{Условие:}}

Висмутовый шарик радиусом 
$ R = 1 $ см помещен в однородное магнитное поле 
$ В = 0,5 $ Тл. Определить магнитный момент 
$ pm $, приобретенный шариком, 
если магнитная восприимчивость $ \chi $ висмута равна 
$ -1,5 * 10^{-4} $.

\underline{\textbf{Решение:}}\\

$ I = \dfrac{p_m}{V} = \dfrac{3p_m}{4\pi R^3} $\\

$ I = \chi H = \chi \dfrac{B_0}{\mu_0\mu} $\\

$ \mu = 1 + \chi $\\

$ \dfrac{3p_m}{4\pi R^3} = \dfrac{B_0\chi}{\mu_0(\chi + 1)} $\\

$ p_m = \dfrac{B_0 \chi 4 \pi R^3}{3\mu_0(\chi + 1)} $\\

$ p_m = 
\dfrac
{0,5 * -1,5 * 10^{-4}  4 \pi 0,01^3}
{3 * 4 \pi * 10^{-7} (-1,5 * 10^{-4}  + 1)} =
- 2.5 * 10^{-4} 
$ А*м$^2$\\


\underline{\textbf{Ответ:}}
$ -250 $ мкА$*$м$^2$\\
