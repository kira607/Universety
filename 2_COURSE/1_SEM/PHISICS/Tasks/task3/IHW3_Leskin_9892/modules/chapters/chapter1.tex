1. Определить силу взаимодействия двух точечных зарядов q1 = q2 = 1 Кл, находящихся в вакууме на расстоянии r = 1 м друг от друга.

1. Два одноименных заряда по 25 нКл каждый закреплены на расстоя-нии r = 24 см друг от друга. Найти напряженность Е электрического поля в точке, удаленной на 15 см от каждого заряда.

1. Найти потенциальную энергию Wп системы трех точечных заря-дов q1 = 10 нКл, q2 = 20 нКл и q3 = – 30 нКл, расположенных в вершинах равностороннего треугольника с длиной стороны а = 10 см.

1. Отрезок тонкого прямого проводника равномерно заряжен с линейной плотностью $ \tau $ = 10 нКл/м. Вычислить потенциал $ \phi $, создаваемый этим заря-дом в точке, расположенной на оси проводника и удаленной от ближайшего конца отрезка на расстояние, равное длине этого отрезка.

1. Плоский конденсатор емкостью 20 пФ соединяют последовательно с таким же конденсатором, но заполненным диэлектриком с диэлектрической проницаемостью $ \varepsilon $ = 3. Найти суммарную емкость С (в пФ) такой батареи.

1. Плоский воздушный конденсатор электроемкостью С = 1,11 нФ заря-жен до разности потенциалов U = 300 В. После отключения от источника то-ка расстояние между пластинами конденсатора было увеличено в 5 раз. Определить: разность потенциалов U на обкладках конденсатора после их раздвижения и работу А внешних сил по раздвижению пластин.

1. Вольтметр, включенный в сеть последовательно с сопротивлением R1, показал напряжение U1 = 198 В, а при включении последовательно с сопро-тивлением R2 = 2R1 показал напряжение U2 = 180 В. Определить сопротивле-ние R1 и напряжение в сети, если сопротивление вольтметра r = 900 Ом.

1. Два источника тока с ЭДС $ \varepsilon $1 = 2 В и $ \varepsilon $2 = 1,5 В и внутренними сопро-тивлениями r1 = 0,5 Ом и r2 = 0,4 Ом включены параллельно сопротивле-нию R = 2 Ом. Определить силу тока через это сопротивление

1. Сила тока в проводнике сопротивлением R = 120 Ом равномерно воз-растает от I0 = 0 до Imax = 5 A за время t = 15 с. Определить выделившееся за это время количество теплоты в проводнике.

1. Напряженность Н0 магнитного поля в вакууме равна 79,6 нА/м. Опре-делить магнитную индукцию В0 этого поля.

9. По контуру в виде квадрата со стороной а = 20 см идет ток I = 50 A. Определить магнитную индукцию В в точке пересечения диагоналей квадрата.

1. В однородном магнитном поле перпендикулярно линиям магнитной ин-дукции В = 1 Тл расположен прямой провод, по которому течет ток I = 1 кА. С какой силой F действует поле на отрезок провода длиной l = 1 м?

1. Найти магнитный поток Ф, создаваемый соленоидом сечения S = 10 см2, если он имеет n = 10 витков на каждый сантиметр его длины при силе тока I = 20 А.

1. Во сколько раз изменилась объемная плотность энергии w магнитного поля тороида со стальным сердечником при изменении магнитной индукции В от 0,5 Тл до 1 Тл и напряженности Н магнитного поля от 220 А/м до 700 А/м, соответственно.

1. Висмутовый шарик радиусом R = 1 см помещен в однородное магнит-ное поле В = 0,5 Тл. Определить магнитный момент pm, приобретенный ша-риком, если магнитная восприимчивость $ \chi $ висмута равна – 1,5·10–4.