\section*{Исследуемые закономерности}

В лабораторной работе измеряется магнитное поле земли 
на основе явления электромагнитной индукции.

При повороте контура, состоящего из $ N $ витков, в однородном магнитном поле с
индукцией $ В $ в нем наводится электродвижущая сила (ЭДС) электромагнитной индукции 

\begin{equation}
	E_i = -\dfrac{d\Psi}{dt}
\end{equation}

$ \Psi = N\Phi $ -- полный магнитный поток (потокосцепление), сцепленный с контуром;

$ \Phi = BS\cos{\alpha} $ -- поток вектора В через плоскую поверхность площадью S,
охватываемую контуром;

$ S = Sn $ -- вектор, равный $ S $ по модулю и направленный по
нормали к этой поверхности;

$ n $ -- орта нормали;

$ \alpha $ -- угол между векторами $ В $ и $ n $.
 
В работе устоновка расположена так, что $ \alpha = 0 $.

ЭДС, возникающая при повороте контура вызывает индукционный ток, переносящий заряд 
через поперечное сечение проводников контура.

В итоге получаем напряжение

\begin{equation}\label{key}
	U = \dfrac{2NBS}{RC}
\end{equation}