\section*{Обработка результатов измерений}

Вычислить средние значения и доверительные погрешности напряжений $U_B$ и $U_{\Gamma}$ и
значений $ B_B $, $ B_\Gamma $, $ B_0 $. Параметры $ R, S, N $ и $ С $ указаны на панели установки.
\ \\ 

Рассчитаем среднее значение и доверительную погрешность $ U_\Gamma $:\\

% выборочное среднее
$ 
\overline{U_{\Gamma}}= 
\sum_{i=1}^{10} U_{\Gamma i} = 
\dfrac{2.3 + 2.3 + 2.3 + 2.5 + 2.5 + 2.5 + 2.6 + 2.6 + 2.6 + 2.7}{10} 
= 2.49
$
\\

% выборочное СКО
$
S_{\overline U_B} = 
\sqrt{
	\dfrac
	{
		\sum_{i=1}^{10}(U_{\Gamma i} - \overline{U_{\Gamma}})
	}
	{
		N(N-1)
	}
}
=\\
\sqrt{
	\dfrac
	{
		0.0361 + 0.0361 + 0.0361 + 0.0001 + 0.0001 + 0.0001 + 0.0121 + 0.0121 + 0.0121 + 0.0441
	}
	{
		90
	}
}
=\\
0.045825756949558406
$
\\

% случайная погрешность
$ 
\varDelta U_{\Gamma} = 
t_{P,N}S_{\overline{U_{\Gamma}}} = 
2.2622 * 0.045825756949558406 = 
0.10366702737129102
$
\\

% полная погрешность
$ 
\varDelta \overline{U_{\Gamma}} = 
\sqrt{\varDelta U_{\Gamma}^2 + \theta_{U_{\Gamma}}^2} =
\sqrt{0.10366702737129102^2 + 0.01^2} = 
0.10414822400790136 \approx 
0.1
$
\\

% результат
$ U_{\Gamma} = 
\overline{U_{\Gamma}} \pm \varDelta \overline{U_{\Gamma}} = 
2.49 \pm 0.1, P = 95\%
$
\\

%########################################################################
\ \\ 

Рассчитаем среднее значение и доверительную погрешность $ B_\Gamma $:\\

$ B_\Gamma = \dfrac{U_\Gamma RC}{2NS} = \dfrac{U_\Gamma RC}{2N\pi r^2} $
\\

$ B_\Gamma = U_\Gamma \dfrac{470 * (2 * 10^{-6})}{2 * 2000 * (3.14 * 0.1^2)} \approx U_\Gamma * 7.48 * 10^{-5} $
\\

\begin{tabular}{|c|c|c|c|c|c|}
	\hline
	$ i $        & 1 & 2 & 3 & 4 & 5 \\
	\hline
	$ B_\Gamma $ & $ 1.72 * 10^{-5} $ & $ 1.72 * 10^{-5} $ & $ 1.72 * 10^{-5} $ & $ 1.87 * 10^{-5} $ & $ 1.87 * 10^{-5} $ \\
	\hline
	$ i $        & 6 & 7 & 8 & 9 & 10 \\
	\hline
	$ B_\Gamma $ & $ 1.87 * 10^{-5} $ & $ 1.94 * 10^{-5} $ & $ 1.94 * 10^{-5} $ & $ 1.94 * 10^{-5} $ & $ 2.02 * 10^{-5} $ \\
	\hline
\end{tabular}
\\ 
\\

% выборочное среднее
$ 
\overline{B_{\Gamma}}= 
\sum_{i=1}^{10} B_{\Gamma i} = 
\dfrac{1.72 * 10^{-5}}{10} + 
\dfrac{1.72 * 10^{-5}}{10} + 
\dfrac{1.72 * 10^{-5}}{10} + 
\dfrac{1.87 * 10^{-5}}{10} + 
\dfrac{1.87 * 10^{-5}}{10} + 
\dfrac{1.87 * 10^{-5}}{10} + 
\dfrac{1.94 * 10^{-5}}{10} + 
\dfrac{1.94 * 10^{-5}}{10} + 
\dfrac{1.94 * 10^{-5}}{10} + 
\dfrac{2.02 * 10^{-5}}{10} = 
\dfrac{18.61 * 10^{-5}}{10} = 
1.861 * 10^{-5}
$
\\

% выборочное СКО
$
S_{\overline U_B} = 
\sqrt{
	\dfrac
	{
		\sum_{i=1}^{10}(B_{\Gamma i} - \overline{B_{\Gamma}})
	}
	{
		N(N-1)
	}
}
=\\
\sqrt{
	\dfrac
	{
		0.0 + 0.0 + 0.0 + 0.0 + 0.0 + 0.0 + 0.0 + 0.0 + 0.0 + 0.0
	}
	{
		90
	}
}
=
\\
\\
0.0
$
\\

Так как значение СКО очень мало, погрешностью можно пренебречь.
\\

% результат
$ B_{\Gamma} = 
\overline{B_{\Gamma}} \pm \varDelta \overline{B_{\Gamma}} = 
1.861 * 10^{-5} \pm 0 \ T, P = 95\%
$
\\

%########################################################################

Рассчитаем среднее значение и доверительную погрешность $ U_B $:\\

% выборочное среднее
$ 
\overline{U_B}= 
\sum_{i=1}^{10} U_{Bi} = 
\dfrac{7.5 + 7.2 + 7.5 + 7.4 + 6.9 + 7.2 + 7.1 + 7.2 + 7.0 + 7.4}{10} 
= 7.24
$
\\

% выборочное СКО
$
S_{\overline U_B} = 
\sqrt{
	\dfrac
	{
		\sum_{i=1}^{10}(U_{Bi} - \overline{U_B})
	}
	{
		N(N-1)
	}
}
=\\
\sqrt{
	\dfrac
	{
		0.0676 + 0.0016 + 0.0676 + 0.0256 + 0.1156 + 0.0016 + 0.0196 + 0.0016 + 0.0576 + 0.0256
	}
	{
		90
	}
}
=\\
0.06531972647421808
$
\\

% случайная погрешность
$ 
\varDelta U_B = 
t_{P,N}S_{\overline{U_B}} = 
2.2622 * 0.06531972647421808 = 
0.14776628522997612
$
\\

% полная погрешность
$ 
\varDelta \overline{U_B} = 
\sqrt{\varDelta U_B^2 + \theta_{U_B}^2} =
\sqrt{0.14776628522997612^2 + 0.01^2} = 
0.14810427087247233 \approx 
0.15
$
\\

% результат
$ U_B = 
\overline{U_B} \pm \varDelta \overline{U_B} = 
7.24 \pm 0.15, P = 95\%
$
\\

%########################################################################
\ \\ 

Рассчитаем среднее значение и доверительную погрешность $ B_B $:\\

$
B_B = 
\dfrac{U_B RC}{2NS} = 
\dfrac{U_B RC}{2N\pi r^2} 
$
\\

$
B_B = 
U_B \dfrac{470 * (2 * 10^{-6})}{2 * 2000 * (3.14 * 0.1^2)} \approx 
U_B * 7.48 * 10^{-6} 
$
\\

\begin{tabular}{|c|c|c|c|c|c|}
	\hline
	$ i $        & 1 & 2 & 3 & 4 & 5 \\
	\hline
	$ B_B $ & $ 5.61 * 10^{-5} $ & $ 5.39 * 10^{-5} $ & $ 5.61 * 10^{-5} $ & $ 5.54 * 10^{-5} $ & $ 5.16 * 10^{-5} $ \\
	\hline
	$ i $        & 6 & 7 & 8 & 9 & 10 \\
	\hline
	$ B_B $ & $ 5.39 * 10^{-5} $ & $ 5.31 * 10^{-5} $ & $ 5.39 * 10^{-5} $ & $ 5.24 * 10^{-5} $ & $ 5.54 * 10^{-5} $  \\
	\hline
\end{tabular}
\\
\\

% выборочное среднее
$ 
\overline{B_{B}}= 
\sum_{i=1}^{10} B_{Bi} = 
\dfrac{5.61 * 10^{-5}}{10} +
\dfrac{5.39 * 10^{-5}}{10} +
\dfrac{5.61 * 10^{-5}}{10} +
\dfrac{5.54 * 10^{-5}}{10} +
\dfrac{5.16 * 10^{-5}}{10} +
\dfrac{5.39 * 10^{-5}}{10} +
\dfrac{5.31 * 10^{-5}}{10} +
\dfrac{5.39 * 10^{-5}}{10} +
\dfrac{5.24 * 10^{-5}}{10} +
\dfrac{5.54 * 10^{-5}}{10} =
\dfrac{54.18 * 10^{-5}}{10} =
5.418 * 10^{-5}
$
\\

% выборочное СКО
$
S_{\overline U_B} = 
\sqrt{
	\dfrac
	{
		\sum_{i=1}^{10}(B_{Bi} - \overline{B_{B}})
	}
	{
		N(N-1)
	}
}
=\\
\sqrt{
	\dfrac
	{
		0.0 + 0.0 + 0.0 + 0.0 + 0.0 + 0.0 + 0.0 + 0.0 + 0.0 + 0.0
	}
	{
		90
	}
}
=\\
\\
0.0
$
\\

Так как значение СКО очень мало, погрешностью можно пренебречь.
\\

% результат
$ B_{B} = 
\overline{B_{B}} \pm \varDelta \overline{B_{B}} = 
5.42 * 10^{-5} \pm 0 \ T, P = 95\%
$
\\

%########################################################################

Рассчитаем $ B_0 $:\\

$ 
B_0^2 = 
B_\Gamma^2 + B_B^2 
$
\\

$ 
B_0 = 
\sqrt{B_\Gamma^2 + B_B^2}
$
\\

$ 
B_0 = 
\sqrt{1.861^2 + 5.418^2} = 
5.729 * 10^{-5} \ T
$
\\

% 5. Рассчитать значения и построить графики ЭДС индукции Ei1(t) и Ei2(t),
% возникающей в катушке при ее равномерном повороте на 180° за время поворота t1 и t2
% вокруг горизонтальной оси (или, по указанию преподавателя, при повороте рамки с
% катушкой вокруг вертикальной оси), используя полученные в работе результаты и
% указанные на панели установки значения t1 и t2. 

