\section*{Исследуемые закономерности}

\textbf{Модель электростатического поля}

В проводящей среде под
действием приложенной к электродам постоянной разности потенциалов
происходит направленное движение заряженных частиц, в результате чего в
среде, окружающей электроды, устанавливается стационарное распределение
потенциала, подобное распределению потенциала в диэлектрической среде
вокруг заряженных проводящих тел, если форма и взаимное расположение
последних аналогичны соответствующим параметрам электродов
проводящей модели. 



\begin{equation}
	F = qE = -q\dfrac{\varDelta \varphi}{\varDelta l}n,
\end{equation}

где $ n $ -- единичный вектор в направлении максимального изменения
потенциала, то в проводящей среде вектор плотности тока подчиняется
вполне симметричному соотношению 

\begin{equation}
	j = -\gamma\dfrac{\Delta\varphi}{\Delta l}n = \gamma E,
\end{equation} 

где $ \gamma $ -- электропроводность среды (величина, обратная удельному
сопротивлению). 


Из сопоставления двух соотношений видно, что:

\begin{enumerate}
	\item оба поля
	потенциальны, (не образуют вихрей в пространстве, окружающем
	электроды);
	\item как линии напряженности электростатического
	поля, так и линии тока перпендикулярны линиям или поверхностям равного
	потенциала. 
\end{enumerate} 

\textbf{Поле длинной двухпроводной линии}

На планшете моделируются поля,
картина которых остается неизменной при
параллельном переносе плоскости, в
которой исследуется поле. 

В данной работе исследуется поле двух
длинных, параллельных, равномерно и
разноименно заряженных проводящих
цилиндров (двухпроводной линии). 

Для каждого цилиндра напряженность поля равна 

\begin{equation}
	E = \dfrac{\tau}{2\pi\varepsilon\varepsilon_0r}
\end{equation}

Значение и направление результирующего вектора напряженности поля
определяют по отношению к системе координат $ x0y $, заданной
экспериментатором. 

\textbf{Напряженность поля и вектор индукции}

Для электростатического
поля справедливо следующее соотношение между вектором напряженности
поля и вектором электрической индукции

\begin{equation}\label{vec}
	D = \varepsilon\varepsilon_0E
\end{equation}

\textbf{Поток вектора индукции электрического поля (теорема Гаусса)}

Поток вектора индукции электрического поля определяется выражением

\begin{equation}
	\Phi_D = \int_S Dds = \int_S Dnds = \int_S Dds\cos(Dn) = \int_S D_n ds
\end{equation}

где $ S $ – поверхность произвольной формы в области поля; $ n $ – единичный
вектор нормали в данной точке поверхности. 

Для электростатического поля справедлива теорема Гаусса

\begin{equation}
	\oint_S Dds = \int_V \rho dV = Q_V
\end{equation}

где $ S $ – произвольная замкнутая поверхность в области поля; $ V $ – объем
области поля, ограниченный поверхностью $ S $; $ Q_V $ – заряд, распределенный в
объеме $ V $. 


Это означает, что выражение (\ref{vec}) следует понимать так: \textit{поток вектора
индукции электростатического поля через замкнутую поверхность
произвольной формы равен суммарному заряду, заключенному в объеме,
ограниченном этой поверхностью, и не зависит от зарядов, расположенных
вне данной поверхности. 
}

\textbf{Циркуляция вектора напряженности электрического поля}

В электрическом поле циркуляцией вектора напряженности
называют физическую величину, которая определяется соотношением 

\begin{equation}\label{key}
	\Gamma = \oint_L Edl = \oint_L Edl\cos(E\tau) = \oint_L E_l dl
\end{equation}

где $ L $ – произвольный замкнутый контур; $ \tau $ – единичный вектор касательной
к линии контура в данной точке. 