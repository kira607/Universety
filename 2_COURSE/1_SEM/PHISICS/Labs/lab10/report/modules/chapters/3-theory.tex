\section*{Исследуемые закономерности}

Ток в цепи $ I $, создаваемый источником ЭДС $ E $ с внутренним сопротивлением $ R_i $
и нагруженный на сопротивление $ R_1 $ рассчитывается как 

\begin{equation}
	I = \dfrac{E}{R_1 + R_i}
\end{equation}

Полная мощность $ P = EI $, развиваемая источником, делится между
нагрузкой и источником следующим образом: 


\begin{equation}
    \dfrac{P_e}{P} = \dfrac{U_e}{U} = \dfrac{R_1}{R_1 + R_i} = \eta
\end{equation}

\begin{equation}
	\dfrac{P_i}{P} = \dfrac{U_i}{U} = \dfrac{R_i}{R_1 + R_i} = 1 - \eta
\end{equation}

где $ P_e = IU_e $ – мощность, выделяющаяся в нагрузке (полезная); $ P_i = IU_i $  –
мощность, выделяющаяся на внутреннем сопротивлении источника;
$ U_e $ и $ U_i $ – падения напряжения на нагрузке и на внутреннем сопротивлении
источника соответственно; $ \eta $ -- КПД источника. 

Напряжение $ U_e $ возрастает
от нуля до значения, равного ЭДС
с увеличением внутреннего сопротивления от нуля (короткое
замыкание) до бесконечности (разомкнутая цепь).

Ток в цепи уменьшается от $ I_{K3} = \dfrac{E}{R_i} $
при коротком замыкании до нуля при разомкнутой цепи.

$ P_e = 0 $ как при коротком замыкании, так и при
разомкнутой цепи.

Максимальная полезная мощность $ P_{e\ max} $ достигается,
когда $ R_1 = R_i $, при так называемом согласовании сопротивлений источника и
нагрузки. В этом случае 


\begin{equation}
	P_{e\ max} = \dfrac{E^2}{4R_i}
\end{equation}

Мощность $ P_{K3} $ развиваемая источником в режиме короткого замыкания составляет:

\begin{equation}
	 P_{K3} = \dfrac{E^2}{R_i}
\end{equation}

C увеличением сопротивления нагрузки полная мощность $ P $ уменьшается и в режиме согласования составляет 

\begin{equation}
	P = \dfrac{E^2}{2R_i}
\end{equation}

Напряжение $ U_e $ в режиме согласования равно половине ЭДС $ E $. КПД
источника равен нулю при коротком замыкании и единице при разомкнутой
цепи; в согласованном режиме $ \eta $ = 0.5.

 