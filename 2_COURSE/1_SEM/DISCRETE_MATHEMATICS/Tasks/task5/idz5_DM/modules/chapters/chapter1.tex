Лескин К.А.

гр. 9892

\section*{Задание}

а) Дан двоичный код длины m. Найти число k дополнительных бит и код Хемминга, соответствующий исходному двоичному коду

б) Дан двоичный код Хемминга. Установить, в каком бите произошла ошибка и восстановить исходный код.
\\

\textbf{Вар. 4}

а) 1011111111

б) 0100010

\section*{Решение а)}

а) 1011111111

$ m = 10 $ --- длина исходного кода.

Найдём $ k $ --- количество контрольных бит:

$ 2^k \geq k + m + 1 $\\

$ 2^k \geq k + 11 $\\

$ k = 4 $\\

Общая длина кода Хемминга $ l $:

$ l = k + m = 14 $\\

Составим таблицу битов и заполним её:\\

\begin{tabular}{|c|c|c|c|c|c|c|c|c|c|c|c|c|c|}
    \hline
    $ * $ & $ * $ & 1 & $ * $ & 0 & 1 & 1 & $ * $ & 1 & 1 & 1 & 1 & 1 & 1 \\
    \hline
    $ b_{1} $ & $ b_{2} $ & $ b_{3} $ & $ b_{4} $ & $ b_{5} $ & $ b_{6} $ & $ b_{7} $ & $ b_{8} $ & $ b_{9} $ & $ b_{10} $ & $ b_{11} $ & $ b_{12} $ & $ b_{13} $ & $ b_{14} $\\
    \hline
\end{tabular}
\\

Выпишем "рабочие"\ множества $ L_{0-3} $:\\

$ L_0 = \{\textbf{1}, 3, 5, 7, 9, 11, 13\} $\\

$ L_1 = \{\textbf{2}, 3, 6, 7, 10, 11, 14\} $\\

$ L_2 = \{\textbf{4}, 5, 6, 7, 12, 13, 14\} $\\

$ L_3 = \{\textbf{8}, 9, 10, 11, 12, 13, 14\}$\\

С помощью суммы Жегалкина найдём контрольные биты $ b_1, b_2, b_4, b_8 $:\\

$ b_1 = b_3 \oplus b_5 \oplus b_7 \oplus b_9 \oplus b_{11} \oplus b_{13} = 
1 \oplus 0 \oplus 1 \oplus 1 \oplus 1 \oplus 1 = 1 $\\

$ b_2 = b_3 \oplus b_6 \oplus b_7 \oplus b_{10} \oplus b_{11} \oplus b_{14} = 
1 \oplus 1 \oplus 1 \oplus 1 \oplus 1 \oplus 1 = 0 $\\

$ b_4 = b_5 \oplus b_6 \oplus b_7 \oplus b_{12} \oplus b_{13} \oplus b_{14} = 
0 \oplus 1 \oplus 1 \oplus 1 \oplus 1 \oplus 1 = 1 $\\

$ b_8 = b_9 \oplus b_{10} \oplus b_{11} \oplus b_{12} \oplus b_{13} \oplus b_{14} = 
1 \oplus 1 \oplus 1 \oplus 1 \oplus 1 \oplus 1 = 0$\\

Запишем результат:

\begin{tabular}{|c|c|c|c|c|c|c|c|c|c|c|c|c|c|}
    \hline
    \textbf{1} & \textbf{0} & 1 & \textbf{1} & 0 & 1 & 1 & \textbf{0} & 1 & 1 & 1 & 1 & 1 & 1 \\
    \hline
    $ b_{1} $ & $ b_{2} $ & $ b_{3} $ & $ b_{4} $ & $ b_{5} $ & $ b_{6} $ & $ b_{7} $ & $ b_{8} $ & $ b_{9} $ & $ b_{10} $ & $ b_{11} $ & $ b_{12} $ & $ b_{13} $ & $ b_{14} $\\
    \hline
\end{tabular}
\\
\\

\textbf{\underline{Ответ:}} 
код Хемминга для исходного сообщения $ \beta =  10110110111111 \\ $


\section*{Решение б)}

б) 0100010

$ l = 7 $\\

Запишем код в виде таблицы:

\begin{tabular}{|c|c|c|c|c|c|c|}
    \hline
    \textbf{0} & \textbf{1} & 0 & \textbf{0} & 0 & 1 & 0 \\
    \hline
    $ b_{1} $ & $ b_{2} $ & $ b_{3} $ & $ b_{4} $ & $ b_{5} $ & $ b_{6} $ & $ b_{7} $\\
    \hline
\end{tabular}
\\
\\

Выпишем "рабочие"\ множества $ L_{0-2} $:\\

$ L_0 = \{\textbf{1}, 3, 5, 7\} $\\

$ L_1 = \{\textbf{2}, 3, 6, 7\} $\\

$ L_2 = \{\textbf{4}, 5, 6, 7\} $\\

Найдём $ U_{0-2} $:\\

$ U_0 = b_1 \oplus b_3 \oplus b_5 \oplus b_7 = 
0 \oplus 0 \oplus 0 \oplus 0 = 0$\\
 
$ U_1 = b_2 \oplus b_3 \oplus b_6 \oplus b_7 = 
1 \oplus 0 \oplus 1 \oplus 0 = 0$\\

$ U_2 = b_4 \oplus b_5 \oplus b_6 \oplus b_7 =
0 \oplus 0 \oplus 1 \oplus 0 = 1$\\

Номер бита $ n = U_2 U_1 U_0 = 100_2 = 4_{10} $\\

Номер $ n = 4 $ совпадает с контрольным битом, поэтому просто вычёркиваем контрольные биты и получаем ответ:\\

\begin{tabular}{|c|c|c|c|c|c|c|}
    \hline
    \sout{0} & \sout{1} & 0 & \textbf{\sout{0}} & 0 & 1 & 0 \\
    \hline
    $ b_{1} $ & $ b_{2} $ & $ b_{3} $ & $ b_{4} $ & $ b_{5} $ & $ b_{6} $ & $ b_{7} $\\
    \hline
\end{tabular}
\\
\\

\textbf{\underline{Ответ:}} 
$ \alpha = 0010 $